\documentclass[11pt,a4paper,sans]{moderncv}
\nopagenumbers{}
\moderncvicons{marvosym} 
\usepackage[scale=0.8]{geometry}
\usepackage[utf8]{inputenc}
\moderncvtheme[blue]{classic}
\moderncvstyle{classic}
\moderncvstyle{oldstyle}
\moderncvstyle{banking}
\moderncvcolor{orange}
\usepackage[top=1.1cm, bottom=1.1cm, left=2cm, right=2cm]{geometry}
\setlength{\hintscolumnwidth}{2.5cm}

\firstname{Clémence}
\familyname{Réda}
\title{}
\address{}{61, avenue du Président Wilson, 94230, Cachan, France\\}
\email{creda@ens-paris-saclay.fr\\}
\homepage{clreda.github.io\\}
%\extrainfo{Car license\\}
%\extrainfo{}
%\phone{00 00 00 00 00}
%\fax{00 00 00 00 00}
%\photo[64pt][0.4pt]{votre_photo}
%\quote{une citation}

\begin{document}
\makecvtitle

\section{Education and qualifications}
\cventry{2017 -- 2018}{Internship}{Faculty of Mathematics, Informatics and Mechanics (University of Warsaw)}{Poland}{}{Ten-month-long internship in research, included in the curriculum of ENS Paris-Saclay (former Ecole Normale Supérieure* de Cachan).\\}
\cventry{2016 -- 2017}{Masters}{ENS Paris-Saclay (former Ecole Normale Supérieure* de Cachan)}{France}{}{First year of Masters diploma in Computer Science, majoring in Bioinformatics, Randomized Algorithms, Convex Optimization, Computer Vision, Machine Learning, Networks, Biology.\\}
\cventry{2015 -- 2016}{Bachelor's degree}{Ecole Normale Supérieure de Cachan and Université Paris Diderot}{France}{}{Bachelor's degree in Computer Science, majoring in Algorithmics, Formal Langages, Programmation, Logics, Architecture and Systems, Computability, Complexity, Databases, Lambda-Calculus, Discrete Mathematics.\\}
\cventry{2013 -- 2015}{\textit{Classe préparatoire aux grandes écoles}**}{Lycée Louis-le-Grand, Paris}{France}{}{Majoring in Mathematics, Computer Science and Physics.\\}
\cventry{2013}{\textit{Baccalauréat S-Mathématiques}***}{Lycée Albert Einstein, Bagnols-sur-Cèze}{France}{}{Majoring in Mathematics.}{\\}
* An \textit{Ecole Normale Supérieure} is a school especially aimed at training future teachers and researchers.\\
** A \textit{classe préparatoire} is a prep course to train undergraduate students for enrollment in highly-selective French colleges. It is considered equivalent to the first two years of a Bachelor's degree.\\
*** The \textit{baccalauréat} is the French equivalent of a final highschool-leaving/A-level examination.\\

\section{Conferences}
\cventry{June \& September 2018}{(\textit{talk})}{BIT'18 and Workshop 6 of ECCB'18}{}{}{C. Réda \& B. Wilczyński. \textbf{Automated Inference of Gene Regulatory Networks Using Explicit Regulatory Modules}.}

\section{Internships and projects}
\cventry{2018}{Internship in Bioinformatics}{Regulomics Team, Dr. B. Wilczy\'{n}ski}{Warsaw, Poland}{10 months}{Theoretical work to include explicit specific biological connections in gene regulatory networks, and study their influence on inference and simulation of such biological models.\\}
\cventry{Early 2017}{Internship in Bioinformatics}{Genomics and Regulatory Systems Unit, Dr. N. Luscombe and Dr. G. Ilsley}{Okinawa, Japan}{5 months}{Implementation in R of an application for single-cell RNA-sequencing data analysis, design and implementation of a novel algorithm for cell clustering.\\}
\cventry{Late 2016}{Design and implementation of a novel recommender system.}{Group project in C++}{}{}{}
\cventry{Summer 2016}{Internship in Bioinformatics}{CBiB team, supervised by Dr. M. Nikolski and M. Raffinot}{Bordeaux, France}{2 months}{Compared analysis of taxonomic trees: improvement of the analysis of metagenomic data in medical studies: \textit{TaxoTree}, \textit{TaxoCluster}, \textit{TaxoClassifier} softwares in Python.\\}
\cventry{2016}{Implementation of a Tower Defense game.}{Group class project in Scala}{}{}{}
\cventry{2016}{Implementation of a parser-lexer with \textit{yacc}.}{Class project in OCaml}{}{}{}
\cventry{Late 2015}{Implementation of a simplified-C compiler.}{Programming class project in OCaml}{}{}{}
\cventry{2013 -- 2015}{Project for the \textit{classe préparatoire}}{With the help of the INSERM of Montpellier, France}{Paris, France}{}{Design and implementation in Python of a model of the evolution of tumoral cells in breast cancer, using circulating tumor DNA.\\}

\section{Skills}
\subsection{Programming languages}
\cvitem{OCaml, Javascript/Typescript, R, Python 2}{Used for several projects, at advanced levels.}
\cvitem{Python 3, Scala, Matlab}{Used for several projects, particularly in bioinformatics.}
\cvitem{LaTeX, TeX, Beamer}{Widely-used for reports and defences.}
\cvitem{Other self-taught languages/frameworks:}{Django, Jekyll, HTML/CSS, PHP, SQL, Bash, C, Intel x86 Assembly, Haskell.}
\subsection{Databases}
\cvitem{MySQL, MongoDB, NeDB}{}
\subsection{Operating Systems}
\cvitem{Windows (up to 10), Linux (Debian, CentOS, Ubuntu), Mac.}{}
\subsection{Softwares}
\cvitem{Adobe Photoshop and Photoshop-like softwares, Microsoft Office, Open Office.}{\\}

\section{Language proficiency}
\cvitemwithcomment{French}{C2}{Native speaker.}
\cvitemwithcomment{English}{C1}{TOEFL score (in 2016): 643/677.}
\cvitemwithcomment{Spanish}{B1}{Intermediate level, studied at school for seven years.}
\cvitemwithcomment{Arabic}{A1}{Basic level, self-taught language.\\}

\section{Miscellaneous}
\cventry{November 2016}{Team wons 27th place out of 60 in the SWERC 2016 contest.}{}{Porto (Portugal)}{}{}
\cventry{2015 -- 2016}{Attended prep courses for the ACM contest.}{}{Paris (France)}{}{}
\cventry{April 2016}{Attended \textit{Django Girls} event.\\\textit{Django Girls} events are meant to teach bases of web programming.}{}{Paris (France)}{}{}
\cventry{2013}{Attended the MASTERCLASS of Physics at the CERN.}{}{Geneva (Switzerland)}{}{}
\cventry{2011}{Won the \textit{Issoire} writing prize.\\The \textit{Issoire} prize is a French regional writing contest.}{}{Nîmes (France)}{}{}

\end{document}
