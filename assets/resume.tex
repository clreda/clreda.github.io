\documentclass[11pt,a4paper,sans]{moderncv}
\nopagenumbers{}
\moderncvicons{marvosym} 
\usepackage[scale=0.8, margin=0.5in]{geometry}
\usepackage[utf8]{inputenc}
\usepackage{fontawesome}
\usepackage{comment}
\usepackage[dvipsnames]{xcolor}
%\moderncvtheme[blue]{classic}
%\moderncvstyle{classic}
%\moderncvstyle{oldstyle}
\moderncvstyle{banking}
\usepackage[unicode]{hyperref}
\moderncvcolor{blue}
\usepackage{CJKutf8}
\setlength{\hintscolumnwidth}{2.5cm}

\firstname{\texttt{\textcolor{blue}{Cl\'{e}mence}}}
\familyname{\texttt{\textcolor{blue}{R\'{e}da}}}
\title{Marie Skłodowska-Curie Postdoctoral Fellow @ Universität Rostock {\tiny (updated on 22/01/2025)}}
\email{clemence.reda@uni-rostock.de}
\homepage{https://clreda.github.io}
%\address{Universität Rostock Lehrstuhl Systembiologie und Bioinformatik}
\address{Systembiologie und Bioinformatik Rostock, Ulmenstraße 69, 18059 Rostock, Germany\\ \\}{}{}
\extrainfo{\faGithub{ \href{https://www.github.com/clreda}{clreda}}}

\begin{document}
\makecvtitle

\vspace{-1.5cm}
%\section{Biography \& Major Achievements}

%The researcher’s research journey led her to consider several aspects of drug development and repurposing, and got her acclimated to working in an interdisciplinary environment. In particular, her PhD was about the integration of a model of gene regulation into a drug repurposing pipeline, in order to properly assess the efficacy of a drug. In order to deal with the intrinsically stochastic regulatory mechanisms, she has designed a recommender system with multi-armed bandit algorithms which can deal with this stochasticity to estimate the most promising drug candidates in a sample-efficient way:
%\begin{enumerate}
%    \item[\textbf{\textcolor{blue}{1.}}] The researcher has worked on a method to automatically build disease-specific regulatory networks. This method allowed to retrieve in a transparent and reproducible way a cell-line-specific network from open-access databases. This method was applied to brain cell lines, on a subset of genes strongly coexpressed with epileptic phenotypes. She introduced for the first time a method which incorporates regulatory cascades and disease-specific transcriptomic information to detect master regulators, which can be targeted for better understanding the mechanisms behind the disease. This work led to a publication in the proceedings of the $20^\textnormal{th}$ \textit{International Conference on Computational Methods in Systems Biology} (\textbf{CMSB} $2022$) and an open-source software which was accepted by the \textitt{Journal of Open-Source Software} (JOSS) in $2023$.
%    \item[\textbf{\textcolor{blue}{2.}}] Multi-armed bandit algorithms were introduced by the researcher for the first time for drug repurposing, by tackling the problem of the identification of the $N$ best arms/treatments in a set of $K$ ones, using the known side information about these arms. In practice, this information is set of the drug signatures built upon transcriptomic data from the LINCS L$1000$ database. She introduced and studied a generic family of algorithms which solve this identification problem, that encompasses all prior adaptive sampling-based algorithms in the literature. Her co-authors and herself were able to study theoretically and empirically the effect of the model structure, especially linear models, on the performance of the algorithms. This work was published in the proceedings of the $24^\textnormal{th}$ \textit{International Conference on Artificial Intelligence and Statistics} (\textbf{AISTATS} $2021$). This problem was tackled for non-linear models in a subsequent work published at the \textit{Advances in Neural Information Processing System Conference} (\textbf{NeurIPS} $2021$, one of the most prestigious conferences in machine learning) with other co-authors.
%    \item[\textbf{\textcolor{blue}{3.}}] She has contributed to the design of a framework integrating non-coding elements in gene regulatory networks. These non-coding elements were shown both theoretically and empirically to effectively structure and constraint the regulatory dynamics across the network, and increase the interpretability and understanding the regulatory mechanisms that lead to a specific transcriptomic phenotype. The corresponding work was published in the \textit{Journal of Theoretical Biology} in $2020$.
%\end{enumerate}
%\vspace{-0.1cm}
%% ADD TOPICS
\section{Professional Experience}
\subsection{Research Positions}

\cventry{From 02/2025}{\textbf{Centre national de la recherche scientifique (CNRS)}, UMR 8197}{Tenured research fellow | \textnormal{Charg\'{e} de recherche de classe normale}}{France}{}{}
\cventry{05/2023--02/2025}{\textbf{Universität Rostock} (SBI Rostock)}{Marie Skłodowska-Curie Postdoctoral Fellow | \textcolor{blue}{\href{https://www.sbi.uni-rostock.de/team/detail/prof-olaf-wolkenhauer}{Pr. Olaf Wolkenhauer}}}{Rostock, Germany}{}{Development of the \textcolor{blue}{\href{https://recess-eu-project.github.io}{RECeSS project}}, focusing on the development of new, improved techniques for drug development based on collaborative filtering approaches.\\\textbf{Skills} Collaborative Filtering · Python (Programming Language) · Applied Machine Learning\\}
\cventry{12/2024--01/2025}{\textbf{Kyoto University} (Machine Learning and Data Mining Research Laboratory)}{MSC PF (research visit) | \textcolor{blue}{\href{https://hkashima.github.io/index_e.html}{Pr. Hisashi Kashima}} \& \textcolor{blue}{\href{https://www.ml.ist.i.kyoto-u.ac.jp/koh.takeuchi/profile/?lang=en}{Dr. Koh Takeuchi}}}{Kyoto, Japan}{}{Joint project on diversity for recommendations in collaborative filtering (part of the \textcolor{blue}{\href{https://recess-eu-project.github.io}{RECeSS project}}).\\\textbf{Skills} Collaborative Filtering · Python (Programming Language) · Applied Machine Learning\\}
\cventry{07/2023--10/2023}{\textbf{Inria Saclay} (SODA team)}{MSC PF (secondment) | \textcolor{blue}{\href{https://jjv.ie/}{Dr. Jill-Jênn Vie}}}{Saclay, France}{}{Design of the \href{https://github.com/recess-eu-project/jeli}{JELI} algorithm, a collaborative filtering approach integrating graph priors to enable explicit interpretability, and application to drug repurposing (part of the \textcolor{blue}{\href{https://recess-eu-project.github.io}{RECeSS project}}).\\\textbf{Skills} Factorization Machines · Knowledge Graph · Interpretability\\}
\cventry{09/2022--03/2023}{\textbf{Neurodiderot} (UMR 1141)}{Postdoctoral position | \textcolor{blue}{\href{http://neurodiderot.org/index.php/en/delahaye-en/}{Pr. Andr\'{e}e Delahaye-Duriez}}}{Paris, France}{}{Development and implementation of the NORDic pipeline for Boolean networks, Prefiguration of the multiomics workflow for the \textcolor{blue}{\href{https://u-paris.fr/recherche-hopitalo-universitaire-rhu/}{RHU FAME}} project led by \textcolor{blue}{\href{https://fr.linkedin.com/in/elie-azoulay-b1315556}{Pr. \'{E}lie Azoulay}}.\\\textbf{Skills} Systems Biology · Programming · Interdisciplinary Research · Bioinformatics\\}
\cventry{09/2019--09/2022 ($36$ months)}{\textbf{Neurodiderot} (UMR 1141) \& \textbf{SCOOL} (UMR 9189)}{PhD position | \textcolor{blue}{\href{http://neurodiderot.org/index.php/en/delahaye-en/}{Pr. Andr\'{e}e Delahaye-Duriez}} \& \textcolor{blue}{\href{http://chercheurs.lille.inria.fr/ekaufman/}{Dr. \'{E}milie Kaufmann}}}{Paris, France}{}{Combination of gene regulatory networks and sequential machine learning for drug repurposing.\\\textbf{Skills} Systems Biology · Multi-Armed Bandits · Interdisciplinary Research · Bioinformatics\\}
\cventry{03/2019--08/2019 ($4$ months)}{\textbf{Neurodiderot} (UMR 1141) \& \textbf{SCOOL} (UMR 9189)}{Master internship | \textcolor{blue}{\href{http://neurodiderot.org/index.php/en/delahaye-en/}{Pr. Andr\'{e}e Delahaye-Duriez}} \& \textcolor{blue}{\href{http://chercheurs.lille.inria.fr/ekaufman/}{Dr. \'{E}milie Kaufmann}}}{Paris, France}{}{Design of a drug repurposing method through a bandit algorithm combined with the prediction of transcriptomic states by a gene regulatory network. Application to the prediction of new anti-epileptics.\\\textbf{Skills} Interdisciplinary Collaboration · Interdisciplinary Research · Statistical Learning · Project Design · Bioinformatics\\}
\cventry{10/2017--07/2018 ($10$ months)}{\textbf{Regulomics team} (MIM UW)}{Predoctoral internship | \textcolor{blue}{\href{http://bioputer.mimuw.edu.pl/~bartek/}{Dr. Bartek Wilczy\'{n}ksi}}}{Warsaw, Poland}{}{Proof-of-concept on the explicit inclusion of biological interactions in gene regulatory networks and its impact on inference and simulation of transcriptomic regulation. Led to a publication in Journal of Theoretical Biology (DOI : \textcolor{blue}{\href{http://doi.org/10.1016/j.jtbi.2019.110091}{10.1016/j.jtbi.2019.110091}}).\\\textbf{Skills} Network Analysis · Epigenetics · Python (Programming Language) · Systems Biology · Scientific Presentation\\}
\cventry{02/2017--07/2017 ($5$ months)}{\textbf{Genomics and Regulatory Systems Unit} (OIST)}{Master internship | \textcolor{blue}{\href{https://www.crick.ac.uk/research/labs/nicholas-luscombe}{Dr. Nicholas Luscombe}} \& \textcolor{blue}{\href{https://www.ebi.ac.uk/about/people/garth-ilsley}{Dr. Garth Ilsley}}}{Onna-son, Japan}{}{Design and implementation of a single-cell RNA sequencing clustering method taking into account intergene expression dependencies using a probabilistic model~; implementation in R Shiny of a web application for the visualisation and preliminary analysis of single-cell RNA sequencing data. Application to transcriptomic data analysis in ciona (\emph{Ciona intestinalis}).\\\textbf{Skills} Benchmarking · R Shiny · Unsupervised Learning · Data Visualization · Python (Programming Language)\\}
\cventry{05/2016--07/2016 ($2$ months)}{\textbf{Centre de Bioinformatique de Bordeaux} (Universit\'{e} de Bordeaux)}{Bachelor internship | \textcolor{blue}{\href{https://www.labri.fr/perso/macha/}{Dr. Macha Nikolski}} \& \textcolor{blue}{\href{https://mabiovis.labri.fr/#team}{Dr. Mathieu Raffinot}}}{Bordeaux, France}{}{Design and implementation of compared analyses of taxonomic trees built from metagenomic data. Application to the analysis of data from intestinal guts of children afflicted with cystic fibrosis at H\^{o}pital Pellegrin in Bordeaux.\\\textbf{Skills} Metagenomics · Phylogenetics · Supervised Learning · Unsupervised Learning · Python (Programming Language)\\}
\subsection{Teaching \& Mentoring Experiences}
%% ADD TOPICS + DEGREE
\cventry{09/2020--09/2021 ($64$ hours)}{Doctorant Contractuel avec Mission d'Enseignement (DCME) (Teaching Assistant)}{Biostatistics, programming and bioinformatics}{Universit\'{e} Paris Cit\'{e}}{}{\textbf{References:} \textcolor{blue}{\href{https://www.researchgate.net/profile/Anne-Badel}{Dr. Anne Badel}} \& \textcolor{blue}{\href{https://www.researchgate.net/profile/Olivier-Taboureau-2}{Pr. Olivier Taboureau}}}
\vspace{0.2cm}
\cventry{08/2024--present}{Joint supervision of Orell Trautmann with \textcolor{blue}{\href{https://www.sbi.uni-rostock.de/team/detail/prof-olaf-wolkenhauer}{Pr. Olaf Wolkenhauer}} (50\%)}{Co-supervision of a PhD}{SBI Rostock}{}{Design of a knowledge graph-based algorithm predicting drug combinations.}
\vspace{0.2cm}
\cventry{09/2023--present}{Joint supervision of Rahul Bordoloi with \textcolor{blue}{\href{https://www.sbi.uni-rostock.de/team/detail/prof-olaf-wolkenhauer}{Pr. Olaf Wolkenhauer}} (33\%)}{Co-supervision of a Masters-equivalent internship and a PhD}{SBI Rostock}{}{Development of a linear discriminant algorithm on multivariate temporal data: \textcolor{blue}{\href{https://arxiv.org/pdf/2402.13103}{paper}}.}
\vspace{0.2cm}
\cventry{02/2020--07/2020 ($6$ months)}{Joint supervision of Adrien Dufour with \textcolor{blue}{\href{http://neurodiderot.org/index.php/en/delahaye-en/}{Pr. Andr\'{e}e Delahaye-Duriez}} (25\%)}{Co-supervision of a Master's degree internship}{Inserm Neurodiderot}{}{Identification of functional families of migroglia cells from targeted single-cell RNA sequencing data of inflammatory microglia at a developmental stage: \textcolor{blue}{\href{https://doi.org/10.1016/j.bbi.2024.09.019}{paper}}.}
\vspace{0.2cm}
\cventry{11/2019--01/2020 ($2$ months)}{Joint supervision of Ariane Alix with \textcolor{blue}{\href{http://chercheurs.lille.inria.fr/ekaufman/}{Dr. \'{E}milie Kaufmann}} (50\%)}{Co-supervision of a Masters's degree project}{ENS Paris-Saclay}{}{Proposal of a project on the adaptation of a published drug-target prediction method to drug repurposing using collaborative filtering in the course \textit{Graphs in Machine Learning} taught by \textcolor{blue}{\href{https://misovalko.github.io/}{Dr. Micha\l{} Valko}} in Master Vision Apprentissage (\textbf{MVA} $2020$).}

\begin{comment}
\subsection{Formations doctorales}
\cvitem{Sept. 2020\\ (3 jours)}{\textbf{FAIR 2020} : ``\textcolor{blue}{\href{https://ressources.france-bioinformatique.fr/fr/evenements/principes-fair-appliques-a-la-bioinformatique}Les principes FAIR appliqués à la bioinformatique}'' \hspace{5cm} (Institut Français de Bioinformatique, IFB)\\}
\cvitem{2019\\(3 mois)}{\textbf{Pr. Walter Fontana} : ``\textcolor{blue}{\href{https://www.college-de-france.fr/site/walter-fontana/course-2019-2020.htm}{La biologie de l'information – un dialogue entre l'informatique et la biologie}}'' (Coll\`{e}ge de France)}
\cvitem{Juil. 2019\\(2 semaines)}{\textbf{RLSS 2019} : ``\textcolor{blue}{\href{https://rlss.inria.fr/}{Reinforcement Learning Summer}}'' (Inria SCOOL)}
\end{comment}

\section{Education}
\cventry{09/2019 -- 09/2022}{PhD in Genetics}{Universit\'{e} Paris Cit\'{e}, Inserm UMR 1141 \& CNRS UMR 9189}{}{}{Doctorate Degree in Science. Title: \textbf{Combination of gene regulatory
networks and sequential machine learning for drug repurposing}, supervised by  \textcolor{blue}{\href{http://neurodiderot.org/index.php/en/delahaye-en/}{Pr. Andr\'{e}e Delahaye-Duriez}} (Inserm UMR 1141) \& \textcolor{blue}{\href{http://chercheurs.lille.inria.fr/ekaufman/}{Dr. \'{E}milie Kaufmann}} (CNRS UMR 9189).\\\textbf{\textit{Viva}: 09/09/2022}.\\}
\cventry{09/2018 -- 09/2019}{M2 Master Vision, Apprentissage (MVA)}{\'{E}cole Normale Supérieure$^\dagger$ (ENS) Paris-Saclay}{(ex-\'{E}cole Normale Supérieure de Cachan)}{}{Master's degree in Machine Learning. (\emph{summa cum laude}, Grade: $16.17/20$, no ranking)\\}
\cventry{09/2016 -- 09/2017}{M1 Master Parisien en Recherche en Informatique (MPRI)}{ENS Paris-Saclay}{}{}{Master's degree in Computer Sciences. (\emph{summa cum laude}, Grade: $16.72/20$, rank: $3/25$)\\}
\cventry{09/2015 -- 09/2016}{L3 Licence informatique fondamentale ENS Cachan}{\'{E}cole Normale Supérieure de Cachan}{}{}{Bachelor's degree in Computer Sciences. (\emph{cum laude}, Grade: $14.64/20$, rank: $10/26$)\\}
$^\dagger$ \small{\textit{\'{E}cole Normale Supérieures are selective French schools for research and teaching.}}

\section{Funding and Awards as Principal Recipient}
\cventry{2024}{PhD award (\textcolor{blue}{\href{ssfam.org/laureats-prix-de-these-ssfam/}{award list}})}{Accessit from the Societe Savante Francophone d'Apprentissage Machine}{SSFAM}{}{}%\\}
\cventry{2023--2025 ($2$ years)}{Postdoctoral grant}{Marie Skłodowska-Curie Postdoctoral Fellowship 2022}{Horizon $2020$}{}{\textcolor{blue}{\href{https://recess-eu-project.github.io}{RECeSS project}}, Project ID: $101102016$.}%\\}
%\cventry{2019--2022 ($3$ years)}{PhD fellowship}{Contrat Doctoral Sp\'{e}cifique aux Normaliens (CDSN) 2019}{French Ministry of Higher Education \& Research}{}{Around $122$ fellowships (less than $50$\% of the candidates) are granted yearly through a selective process based on scholastic records, recommendation letters, and the project proposal.\\}
%\cventry{2017,2018 ($2$ years)}{Scholarship}{\'{E}l\`{e}ve fonctionnaire stagiaire de l’ENS Cachan}{French Ministry of Higher Education \& Research}{}{Intern civil servant student at ENS de Cachan (Second concours: oral admission test in Computer Sciences, rank: 2, 4 positions every year at national level). Caesura for personal convenience from 01/09/2017 to 31/08/2018.\\}

\section{Research}
\subsection{Preprints}
\cventry{}{}{An Anytime Algorithm for Good Arm Identification\\M. Jourdan \& \underline{C. Réda}}{}{Under review, DOI: \textcolor{blue}{\href{https://arxiv.org/pdf/2310.10359.pdf}{ 	
10.48550/arXiv.2310.10359}}}{}{}{}{}

\subsection{Peer-Reviewed Scientific Journals}
\vspace{0.1cm}\textcolor{blue}{\textbf{2025}}\vspace{0.2cm}\\
\cventry{}{}{Multivariate Functional Linear Discriminant Analysis for the Classification of Short Time Series with Missing Data\\R. Bordoloi, \underline{C. Réda}, O. Trautmann, S. Bej, O. Wolkenhauer}{}{In press, Machine Learning, DOI: \textcolor{blue}{\href{https://arxiv.org/abs/2402.13103}{ 	
10.48550/arXiv.2402.13103}}}{}{}{}{}
\cventry{}{}{Comprehensive evaluation of pure and hybrid collaborative filtering in drug repurposing\\\underline{C. Réda}, J.-J. Vie, O. Wolkenhauer}{}{Scientific Reports, 15, 2711, DOI: \textcolor{blue}{\href{https://doi.org/10.1038/s41598-025-85927-x}{ 	
10.1038/s41598-025-85927-x}}}{}{}{}{}
\vspace{0.1cm}\\
\cventry{}{}{Joint Embedding-Classifier Learning for Interpretable Collaborative Filtering\\\underline{C. Réda}, J.-J. Vie, O. Wolkenhauer}{}{BMC Bioinformatics, 26, 26, DOI: \textcolor{blue}{\href{https://doi.org/10.1186/s12859-024-06026-8}{ 	
10.1186/s12859-024-06026-8}}}{}{}{}{}
\vspace{0.1cm}\\
\cventry{}{}{Neonatal inflammation impairs developmentally-associated microglia and promotes a highly reactive microglial subset\\A. Dufour$^*$, A. Heydari-Olya$^*$, S. Foulon$^*$, \underline{C. Réda}, et al.}{}{Brain, Behavior, and Immunity, DOI: \textcolor{blue}{\href{https://doi.org/10.1016/j.bbi.2024.09.019}{10.1016/j.bbi.2024.09.019}}}{}{}{}{}
\vspace{0.1cm}\textcolor{blue}{\textbf{2024}}\vspace{0.2cm}\\
\cventry{}{}{stanscofi and benchscofi: a new standard for drug repurposing by collaborative filtering\\\underline{C. Réda}, J.-J. Vie, O. Wolkenhauer}{}{Journal of Open Source Software, 9(93):5973, DOI: \textcolor{blue}{\href{https://doi.org/10.21105/joss.05973}{10.21105/joss.05973}}}{}{}{}{}
\vspace{0.1cm}\textcolor{blue}{\textbf{2023}}\vspace{0.2cm}\\
\cventry{}{}{NORDic: a Network-Oriented package for the Repurposing of Drugs\\\underline{C. Réda} \& A. Delahaye-Duriez}{}{Journal of Open Source Software, 8(90):5532, DOI: \textcolor{blue}{\href{https://doi.org/10.21105/joss.05532}{10.21105/joss.05532}}}{}{}{}{}
\vspace{0.1cm}\textcolor{blue}{\textbf{2021}}\vspace{0.2cm}\\
\cventry{}{}{Machine learning applications in drug development\\\underline{C. Réda}, \'{E}. Kaufmann \& A. Delahaye-Duriez}{}{Computational and Structural Biotechnology Journal, 18:241-252, DOI: \textcolor{blue}{\href{https://doi.org/10.1016/j.csbj.2019.12.006}{10.1016/j.csbj.2019.12.006}}}{}{}{}{}
\vspace{0.1cm}\textcolor{blue}{\textbf{2020}}\vspace{0.2cm}\\
\cventry{}{}{Automated inference of gene regulatory networks using explicit regulatory modules\\\underline{C. Réda} \& B. Wilczy\'{n}ski}{}{Journal of Theoretical Biology, 486:110091, DOI: \textcolor{blue}{\href{https://doi.org/10.1016/j.jtbi.2019.110091}{10.1016/j.jtbi.2019.110091}}}{}{}{}{}
\vspace{0.1cm}\textcolor{blue}{\textbf{2019}}\vspace{0.2cm}\\
\cventry{}{}{Identification de cibles thérapeutiques et repositionnement de médicaments par analyses de réseaux géniques\\ A. Delahaye-Duriez, \underline{C. Réda} \& P. Gressens}{}{M\'{e}decine/Sciences, 35:515-518, DOI: \textcolor{blue}{\href{https://doi.org/10.1051/medsci/2019108}{10.1051/medsci/2019108}}}{}{}{}{}

\subsection{Peer-Reviewed Conference Proceedings}
\vspace{0.1cm}\textcolor{blue}{\textbf{2022}}\vspace{0.2cm}\\
\cventry{}{}{Near-optimal Collaborative Learning in Bandits\\\underline{C. Réda}, S. Vakili, {\'{E}}. Kaufmann}{}{Proceedings of the $36^{\textnormal{th}}$ Conference on Advances in Neural Information Processing Systems (\textbf{NeurIPS} $2022$)}{}{}{}{\textit{HAL: \textcolor{blue}{\href{https://hal.archives-ouvertes.fr/hal-03825099}{03825099}}} [\textcolor{blue}{\textit{Selected as Oral}}]\\}
\cventry{}{}{Prioritization of Candidate Genes Through Boolean Networks\\\underline{C. Réda}, {A}. Delahaye-Duriez}{}{Proceedings of the $20^{\textnormal{th}}$ International Conference on Computational Methods in Systems Biology (\textbf{CMSB} $2022$)}{}{}{}{Springer:89-121 [\textcolor{blue}{\textit{Best Student Paper Award}}]\\}
\vspace{0.2cm}\textcolor{blue}{\textbf{2021}}\vspace{0.2cm}\\
\cventry{}{}{Dealing With Misspecification In Fixed-Confidence Linear Top-m Identification\\\underline{C. Réda}, {A}. Tirinzoni \& R. Degenne}{}{Proceedings of the $35^{\textnormal{th}}$ Conference on Neural Information Processing Systems (\textbf{NeurIPS} 2021), 34, HAL: \textcolor{blue}{\href{https://hal.archives-ouvertes.fr/hal-03409205}{03409205}}}{}{}{}{}
\cventry{}{}{Top-$m$ identification for linear bandits\\\underline{C. Réda}, \'{E}. Kaufmann \& A. Delahaye-Duriez}{}{Proceedings of the $24^{\textnormal{th}}$ International Conference on Artificial Intelligence and Statistics (\textbf{AISTATS} 2021), 130}{}{}{}{\textit{HAL: \textcolor{blue}{\href{https://hal.archives-ouvertes.fr/hal-03172145}{03172145}}}}
\bigskip

\subsection{Oral Communications at International Conferences}
\cventry{10/10/2023}{e:Med Meeting 2023 on Systems Medicine (Berlin, Germany)}{\underline{C. Réda}. Benchmarking collaborative filtering approaches to drug repurposing}{}{}{}
\cventry{07/12/2022}{$35^\textnormal{th}$ International Conference on Advances in Neural Information Processing Systems (New Orleans, USA)}{\underline{C. Réda}. Near-optimal Collaborative Learning in Bandits}{}{}{}
\cventry{14/09/2022}{$20^\textnormal{th}$ International Conference on Computational Methods in Systems Biology (Bucharest, Romania)}{\underline{C. Réda}. Prioritization of Candidate Genes Through Boolean Networks}{}{}{}
\cventry{10/02/2022}{$13^\textnormal{th}$ Conference on Dynamical Systems Applied to Biology and Natural Sciences (held virtually)}{\underline{C. Réda}. Gene network oriented drug discovery: automated inference of Boolean networks (...)}{}{}{}
\cventry{08/12/2021}{NeurIPS@Paris $2021$ (Paris, France)}{\underline{C. Réda}. Dealing With Misspecification In Fixed-Confidence Linear Top-m Identification}{}{}{}
\cventry{02/07/2020}{Journ\'{e}es Ouvertes de Biologie, Informatique et Math\'{e}matique (JOBIM) 2020 (held virtually)}{\underline{C. Réda}. Automated inference of gene regulatory networks using explicit regulatory modules}{}{}{}
%\cventry{08/09/2018}{Workshop 6 of the 17th European Conference on Computational Biology (ECCB 2018, Athens, Greece)}{\underline{C. Réda}. Automated inference of gene regulatory networks using explicit regulatory modules}{}{}{}
%\cventry{28/06/2018}{BioInformatics in Toruń (BIT 2018, Toruń, Poland)}{\underline{C. Réda}. Automated inference of gene regulatory networks using explicit regulatory modules}{}{}{}

\subsection{Poster Presentations at International Conferences}
\cventry{09/2024}{ECCB 2024 (Turku, Finland)}{\underline{C. Réda}. JELI: an interpretable embedding-learning recommender system for drug repurposing}{}{}{}
\cventry{06/2024}{JOBIM 2024 (Toulouse, France)}{\underline{C. Réda}. JELI: an interpretable embedding-learning recommender system for drug repurposing}{}{}{}
\cventry{02/2024}{SMPGD 2024 (Paris, France)}{\underline{C. Réda}. Towards a large-scale benchmark of collaborative filtering in drug repurposing}{}{}{}
\cventry{07/2023}{ISMB/ECCB $2023$ (Lyon France)}{\underline{C. Réda}. Drug repurposing in breast cancer by combining bandit algorithms and Boolean networks (...)}{}{}{}
\cventry{07/2022}{Journ\'{e}es Ouvertes de Biologie, Informatique et Math\'{e}matique (JOBIM 2022, Rennes, France)}{\underline{C. Réda}. Prioritization of Candidate Genes Through Influence Maximization}{}{}{}
\cventry{12/2021}{$35^\textnormal{th}$ International Conference on Advances in Neural Information Processing Systems (NeurIPS 2022, held virtually)}{\underline{C. Réda}. Dealing With Misspecification In Fixed-Confidence Linear Top-m Identification}{}{}{}
\cventry{04/2021}{$24^\textnormal{th}$ International Conference on Artificial Intelligence and Statistics (AISTATS 2021, held virtually)}{\underline{C. Réda}. Top-$m$ identification for linear bandits}{}{}{}

\section{Open-Source Softwares \& Datasets}
\subsection{Softwares}
\vspace{0.1cm}\textcolor{blue}{\textbf{2024}}\vspace{0.2cm}\\
\cventry{}{}{Joint Embedding-classifier Learning for improved Interpretability (JELI)\\ \underline{C. Réda}}{}{Zenodo, DOI: \textcolor{blue}{\href{https://doi.org/10.5281/zenodo.12193722}{10.5281/zenodo.12193722}}, GitHub: \texttt{recess-eu-project/JELI}}{Python package implementing an explicitly interpretable collaborative filtering}{}{}{}
\vspace{0.1cm}\textcolor{blue}{\textbf{2023}}\vspace{0.2cm}\\
\cventry{}{}{BENCHmark for drug Screening with COllaborative FIltering (benchscofi)\\ \underline{C. Réda}}{}{Zenodo, DOI: \textcolor{blue}{\href{https://doi.org/10.5281/zenodo.8241505}{10.5281/zenodo.8241505}}, GitHub: \texttt{recess-eu-project/benchscofi}}{Python package implementing algorithms and methods from the state-of-the-art in drug repurposing with collaborative filtering}{}{}{}
\cventry{}{}{STANdard for drug Screening by COllaborative FIltering (stanscofi)\\ \underline{C. Réda}}{}{Zenodo, DOI: \textcolor{blue}{\href{https://doi.org/10.5281/zenodo.8038847}{10.5281/zenodo.8038847}}, GitHub: \texttt{recess-eu-project/stanscofi}}{Python package for the automation of the training and validation of drug repurposing with machine learning}{}{}{}
\cventry{}{}{Network Oriented Repurposing of Drugs (NORDic)\\ \underline{C. Réda}}{}{Zenodo, DOI: \textcolor{blue}{\href{https://doi.org/10.5281/zenodo.7239047}{10.5281/zenodo.7239047}}, GitHub: \texttt{clreda/NORDic}}{Python package for the inference, analysis of Boolean networks \& application to drug repurposing}{}{}{}
\subsection{Datasets}
\vspace{0.1cm}\textcolor{blue}{\textbf{2023}}\vspace{0.2cm}\\
\cventry{}{}{PREDICT\\ \underline{C. Réda}}{}{Zenodo, DOI: \textcolor{blue}{\href{https://doi.org/10.5281/zenodo.7982964}{10.5281/zenodo.7982964}}}{Large drug repurposing dataset \textcolor{blue}{\href{https://github.com/RECeSS-EU-Project/drug-repurposing-datasets}{with open-source generation}}}{}{}{}
\cventry{}{}{TRANSCRIPT\\ \underline{C. Réda}}{}{Zenodo, DOI: \textcolor{blue}{\href{https://doi.org/10.5281/zenodo.7982969}{10.5281/zenodo.7982969}}}{Drug repurposing dataset on transcriptomic data \textcolor{blue}{\href{https://github.com/RECeSS-EU-Project/drug-repurposing-datasets}{with open-source generation}}}{}{}{}

\section{Commitment to Popularization of Sciences and Law Making}
\subsection{Popularization of Sciences}
\cvitem{11/21/2024}{Popularization paper (in French) on drug repurposing aimed at medical practitioners: \underline{C. Réda}, B. Villoutreix and A. Delahaye-Duriez. \textbf{Repositionnement de médicaments} \textit{In} La Revue du Praticien, 21 novembre 2024, 74(9);942-6 (\textcolor{blue}{\href{https://www.larevuedupraticien.fr/article/repositionnement-de-medicaments}{link}})}
\cvitem{05/2023--hiatus}{\textbf{Created and published on} \textcolor{blue}{\href{https://recess-eu-project.github.io/}{RECeSS project blog}}: progress reports on the \textcolor{blue}{\href{https://zenodo.org/api/grants/10.13039/501100000780::101102016}{RECeSS project}} and introductory blog posts on drug repurposing and collaborative filtering.}
\cvitem{12/2016--09/2018}{\textbf{Published on} \textcolor{blue}{\href{https://tryalgo.org/fr/}{Tryalgo}} [in French] : series of blog posts on known algorithms with concrete applications, aimed at high school and college students (approx. $2,400$ unique monthly users~; two of these posts constitute the Top-$2$ most visited pages.} 
\cvitem{10/2016}{\textbf{Published on} \textcolor{blue}{\href{https://www.lemonde.fr/blog/binaire/2016/10/17/a-p-b-la-vie-apres-le-bac/}{Binaire}} (blog on Computer Science affiliated with French newspaper \textit{Le Monde}) \textbf{and} \textcolor{blue}{\href{https://theconversation.com/apb-la-vie-apres-le-bac-66848}{The Conversation}} [in French] : ``A.P.B. : La vie après le bac'' (conjointy written with \textcolor{blue}{\href{https://www.inria.fr/en/serge-abiteboul}{Serge Abiteboul}}). Explanation of the algorithm of Gale-Shapley which has been in use in a previous version of the French national web application for high school students' applications to college} 
\subsection{Popularization of Law-Making}
\cvitem{12/2016--09/2018}{\textbf{Published on} \textcolor{blue}{\href{https://reflechir.fr/}{R\'{e}fl\'{e}chir.fr}} [in French] : series of blog posts on laws passed since 2017 in France: explanation of their content and their consequences ($534$ followers on February, $24$ $2021$).} 

%\section{Miscellaneous}
%\cvitem{2023--present}{\textbf{Selection as postdoc female mentoree by Universität Rostock} to develop her professional career: access to seminars and personalized mentoring (see the \textcolor{blue}{\href{https://www.uni-rostock.de/universitaet/vielfalt-und-gleichstellung/karrierewegementoring/postdoktorandinnen-und-juniorprofessorinnen/postdoktorandinnen-jahrgang-2023/}{list of selected candidates}}).}

\end{document}